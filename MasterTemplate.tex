\documentclass[single]{HBUThesis}
% 默认值:公开
\miji{保密}
% 默认值:空
\fenlh{123}
% 默认值:10075
\xxdaima{10075}
\StuNum{20192049}
\major{医学硕士}
\Emajor{Master of Science}
\speciality{基础医学}
\Especiality{Basic Science of Medicine}
\date{二〇二〇年六月}
\Edate{June, 2020}
\supervisor{牛嗣云}
\Esupervisor{Prof. Niu Siyun}
\collageName{基础医学院}
\EuniverName{Hebei University}
% 建议 25 字,(包括副标题和标点符号)。论文题目用语必须准确,应该是对研究对象的
% 准确具体的描述,一般要在一定程度上体现研究结论。论文题目通常由名词性短语构成,
% 应尽量避免使用不常用缩略词、首字母缩写字、字符、代号和公式等 (如论文题目内容层
% 次很多,难以简化时,可采用论文题目和论文副标题相结合的方法,其中副标题起补充、
% 阐明题目的作用)
\title{标题标题标题标题标题标题标题标题标题标题标题标题}
\Etitle{title, title}
\author{宋邑诚}
\Eauthor{Song Yi cheng}
\UDC{}
\begin{document}

% 封面
\makecover
\makeEncover

% 独创性声明,学位评定委员会办公室下载
\orgState

% 正文
你好,世界!
hello, world
\textbf{
  你好,世界!
  hello, world
}

\songti 宋体 \textbf{加粗}

\heiti 黑体 \textbf{加粗}

\fangsong 仿宋 \textbf{加粗}

\kaiti 楷体 \textbf{加粗}

\FZXBSong 方正小标宋 \textbf{加粗}


\end{document}
